%!TEX program = xelatex

\documentclass{uniquecv}

\usepackage{fontawesome5}
\usepackage{fontspec}
\usepackage[colorlinks]{hyperref}
% ----------------------------------------------------------------------------- %

%\setmonofont{FiraCode}
\begin{document}
\name{张严}
\medskip

\basicinfo{
  \faPhone ~ (+86) 181-7165-0707
  \textperiodcentered\
  \faEnvelope ~ Tracyyan1030@gmail.com
  \textperiodcentered\
  \faGithub ~ github.com/Cloudyunn
  \textperiodcentered\
  \faBlog ~ \href{https://cloudyunn.github.io/}{\color{black}{blog}}
}

% ----------------------------------------------------------------------------- %

\section{教育背景}
\dateditem{\textbf{浙江大学} \quad 电子信息 \quad 全日制硕士}{2021年 -- 至今 }
GPA:89.8/100.0  \quad \textbf{国家重点实验室}
\begin{itemize}
  \item 浙江大学学业优秀奖学金
  \item 浙江大学三好研究生、优秀研究生、优秀研究生干部、优秀团干部
  \item 浙江大学NGICS大平台年度优秀研究生一等奖
\end{itemize}

\dateditem{\textbf{武汉科技大学} \quad 自动化 \quad 工学学士}{2017年 -- 2021年 }
GPA:3.82/4.0 \quad 专业排名:1/197 \quad 年级前1\% 
\begin{itemize}
  \item 推荐免试至浙江大学
  \item \textbf{国家奖学金(top 1\% )}、校级特等奖学金、一等奖学金、许家印奖学金
  \item 校级优秀毕业生、优秀学生、模范共青团干部、优秀共青团干部
  \item 全国大学生数学建模湖北省一等奖、全国大学生英语竞赛国家三等奖
\end{itemize}

% ----------------------------------------------------------------------------- %
\section{项目/获奖经历}
\datedproject{网络安全态势感知系统}{课题组项目}{2022年 3月 - 2023年 4月}
{\it 《高端工程装备智能与安全技术研究》 1000万}
\quad \emph{漏洞挖掘、架构设计、后端开发}
\vspace{0.4ex}

\begin{itemize}
  \item 针对生产线不同操作系统AGV小车进行协议层面的漏洞挖掘
  \item 设计网络安全态势感知系统框架、软件架构、数据库
  \item 资产识别/入侵检测/漏洞态势聚合等引擎模块开发
  \item 前端技术栈:Vue/JavaScript/HTML
  \item 后端技术栈:Java/SpringBoot/mysql/redis
\end{itemize}
\quad 成果:申请CNVD原创漏洞证明3项,团队共计完成专利3项,项目交付系统1套

\datedproject{谷粒学苑}{课内项目}{2021年 4月 - 2021年 8月}
{\it }
\vspace{0.4ex}
\begin{itemize}
  \item 前端技术栈:HTML/CSS/JavaScript/Vue.js/Element UI/Webpack
  \item 后端技术栈:Java/Spring Boot/Spring Cloud/MySQL/Redis/MyBatis/Tomcat/ZooKeeper
  \item 代码管理:Git/Maven/Jenkins
\end{itemize}

\datedproject{全国大学生数学建模竞赛}{湖北省一等奖}{2019年 11月 }
{\it Matblab/数学建模}
\quad \emph{负责所有代码编写}
\vspace{0.4ex}
\begin{itemize}
  \item 基于刚体动力学、弹性碰撞和力矩合成等研究,采用迭代算法和空间坐标变换思想建立了同心鼓的运动模型
和弹性碰撞数学模型。
\end{itemize}
\vspace{0.4ex}

% ----------------------------------------------------------------------------- %

\section{专业技能}
\smallskip
\textbf{Programming}
\quad Java/Python/C\quad Linux/shell基本使用

\textbf{JavaSE} 
\quad 熟悉Java基础/JVM/并发编程

\textbf{Software} 
\quad MySQL/Redis

\textbf{Basic} 
\quad 计算机网络/操作系统/数据结构与算法/计算机组成原理

\textbf{框架} 
\quad 熟悉Spring/SpringBoot/MyBatis/SpringCloud

\textbf{语言能力} 
\quad CET4-563 \quad CET6-537
% ----------------------------------------------------------------------------- %

\end{document}