%!TEX program = xelatex

\documentclass{uniquecv}

\usepackage{fontawesome5}
\usepackage{fontspec}
\usepackage[colorlinks]{hyperref}
% ----------------------------------------------------------------------------- %

%\setmonofont{FiraCode}
\begin{document}
\name{张严}
\medskip

\basicinfo{
  \faPhone ~ (+86) 181-7165-0707
  \textperiodcentered\
  \faEnvelope ~ Tracyyan1030@gmail.com
  \textperiodcentered\
  \faGithub ~ github.com/Cloudyunn
  \textperiodcentered\
  \faBlog ~ \href{https://cloudyunn.github.io/}{\color{black}{blog}}
}

% ----------------------------------------------------------------------------- %

\section{教育背景}
\dateditem{\textbf{浙江大学} \quad 电子信息 \quad 硕士}{2021年 -- 至今 }
\dateditem{\textbf{武汉科技大学} \quad 自动化 \quad 本科}{2017年 -- 2021年 }
GPA:3.82/4.0 专业排名:1/197 \quad 年级前1\% \quad 英语:CET4-563 \quad CET6-537
\text推荐免试至浙江大学

% ----------------------------------------------------------------------------- %

\section{专业技能}
\smallskip
\textbf{Programming}
\quad Java/Python/C/Linux

\textbf{Tools} 
\quad \LaTeX/MarkDown/Git/IDEA

\textbf{Software} 
\quad MySQL/Redis

\textbf{Basic} 
\quad 计算机网络/操作系统/数据结构/计算机组成原理

\textbf{JavaSE} 
\quad 熟悉Java语言/面向对象思想/并发编程/JVM/socket编程

\textbf{JavaEE} 
\quad Spring/SpringBoot/MyBatis/SpringMVC/SpringCloud

\textbf{Security} 
\quad WEB/渗透测试/PWN/Misc

% ----------------------------------------------------------------------------- %

\section{项目/科研经历}
\datedproject{网络安全态势感知系统}{课题组项目}{2022年 3月 - 2023年 4月}
{\it 《高端工程装备智能与安全技术研究》 1000万}
\quad \emph{漏洞挖掘、架构设计、后端开发}
\vspace{0.4ex}

\begin{itemize}
  \item 针对生产线不同操作系统AGV小车进行协议层面的漏洞挖掘
  \item 设计网络安全态势感知系统框架、软件架构、数据库
  \item 资产识别/入侵检测/漏洞态势聚合等引擎模块开发
  \item 前端技术栈:Vue/JavaScript/HTML
  \item 后端技术栈:Java/SpringBoot/mysql/redis
\end{itemize}
\quad 成果:申请CNVD原创漏洞证明3项,团队攻击完成专利3项,项目交付系统1套

\datedproject{工控协议逆向与协议漏洞挖掘}{课题组项目}{2021年 9月 - 2021年 12月}
{\it }
\quad \emph{协议逆向、漏洞挖掘、Python/Socket编程}
\vspace{0.4ex}

\begin{itemize}
  \item 负责施耐德电气工业控制私有化通信协议——UMAS协议客户端、服务端的开发,主要面向协议规范伪造、发送以及响应大
量流量数据包,主要是面向Socket的python编程。
\end{itemize}

\datedproject{谷粒学苑}{课内项目}{2021年 4月 - 2021年 8月}
{\it }
\vspace{0.4ex}

\begin{itemize}
  \item 前端技术栈:HTML/CSS/JavaScript/Vue.js/React/Element UI/Webpack
  \item 后端技术栈:Java/Spring Boot/Spring Cloud/MySQL/Redis/MyBatis/Tomcat/ZooKeeper
  \item 代码管理:Git/Maven/Jenkins
\end{itemize}

% ----------------------------------------------------------------------------- %

\section{社团/工作经历}
\datedproject{浙江大学NGICS大平台团支部书记}{}{2021年 9月 - 2022年 9月}
\datedproject{武汉科技大学香涛计划工科班团支部书记}{}{2018年 9月 - 2021年 6月}
\datedproject{湖北省大冶市工商局暑期社会实践}{}{2018年 6月 - 2018年 8月}

% ----------------------------------------------------------------------------- %
\section{荣誉奖项}

\datedproject{CNVD}{}{}

{\it}
\vspace{0.4ex}

\begin{itemize}
  \item \dateditem{\textbf{信捷电气XD5E-24T-E密码泄露漏洞(中危)}}{2022年5月}
  \item \dateditem{\textbf{信捷PLC编程工具软件密码泄露漏洞(中危)}}{2022年5月}
  \item \dateditem{\textbf{信捷电气XD5E-24T-E拒绝服务漏洞(中危)}}{2021年12月}
\end{itemize}

\datedproject{研究生}{}{}
{\it}
\vspace{0.4ex}

\begin{itemize}
  \item \dateditem{\textbf{浙江大学三好研究生}}{2021 -- 2022学年}
  \item \dateditem{\textbf{浙江大学优秀研究生}}{2021 -- 2022学年}
  \item \dateditem{\textbf{浙江大学优秀研究生干部}}{2021 -- 2022学年}
  \item \dateditem{\textbf{浙江大学优秀团干部}}{2021 -- 2022学年}
  \item \dateditem{\textbf{浙江大学优秀团干部}}{2021 -- 2022学年}
  \item \dateditem{\textbf{浙江大学NGICS大平台年度优秀研究生一等奖}}{2021 -- 2022学年}
\end{itemize}

\datedproject{本科}{}{}
{\it}
\vspace{0.4ex}

\begin{itemize}
  \item \dateditem{\textbf{武汉科技大学优秀毕业生}}{2021年6月}
  \item \dateditem{\textbf{武汉科技大学优秀毕业设计}}{2021年6月}
  \item \dateditem{\textbf{国家奖学金}}{2019 -- 2020学年}
  \item \dateditem{\textbf{全国大学生数学建模竞赛湖北省一等奖}}{2019年11月}
  \item \dateditem{\textbf{全国大学生英语竞赛国家三等奖}}{2018年5月}
  \item \dateditem{\textbf{武汉科技大学优秀共青团员、模范共青团干部}}{2018 -- 2020学年}
  \item \dateditem{\textbf{武汉科技大学特等奖学金}}{2018 -- 2019学年}
  \item \dateditem{\textbf{武汉科技大学优秀学生}}{2017 -- 2020学年}
  \item \dateditem{\textbf{武汉科技大学一等奖学金}}{2017 -- 2019学年}
\end{itemize}



\end{document}